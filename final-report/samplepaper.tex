% This is samplepaper.tex, a sample chapter demonstrating the
% LLNCS macro package for Springer Computer Science proceedings;
% Version 2.21 of 2022/01/12
%
\documentclass[runningheads]{llncs}
%
\usepackage[T1]{fontenc}
% T1 fonts will be used to generate the final print and online PDFs,
% so please use T1 fonts in your manuscript whenever possible.
% Other font encondings may result in incorrect characters.
%
\usepackage{graphicx}
% Used for displaying a sample figure. If possible, figure files should
% be included in EPS format.
%
% If you use the hyperref package, please uncomment the following two lines
% to display URLs in blue roman font according to Springer's eBook style:
%\usepackage{color}
%\renewcommand\UrlFont{\color{blue}\rmfamily}
%\urlstyle{rm}
%
\begin{document}
%
\title{HUBSIM: Simulating Human Behavior In Negotiation Settings}
%
%\titlerunning{Abbreviated paper title}
% If the paper title is too long for the running head, you can set
% an abbreviated paper title here
%
\author{Nico Döring \and
Second Author\inst{2,3}\orcidID{1111-2222-3333-4444} \and
Third Author\inst{3}\orcidID{2222--3333-4444-5555} \and Fourth Author}
%
\authorrunning{F. Author et al.}
% First names are abbreviated in the running head.
% If there are more than two authors, 'et al.' is used.
%
\institute{Princeton University, Princeton NJ 08544, USA \and
Springer Heidelberg, Tiergartenstr. 17, 69121 Heidelberg, Germany
\email{lncs@springer.com}\\
\url{http://www.springer.com/gp/computer-science/lncs} \and
ABC Institute, Rupert-Karls-University Heidelberg, Heidelberg, Germany\\
\email{\{abc,lncs\}@uni-heidelberg.de}}
%
\maketitle              % typeset the header of the contribution
%
\begin{abstract}
The abstract should briefly summarize the contents of the paper in
150--250 words.

\keywords{First keyword  \and Second keyword \and Another keyword.}
\end{abstract}
%
%
%
\section{Introduction}

Add an introduction here.


\section{Related Work}

The rapid improvements of recent large language model capabilities prompted an increasing interest in exploring the potential for agent-based experimental studies and simulations in social science research. Park et al. \cite{park_generative_2023}  simulate and explore an entire human society and its behavioral interactions through an LLM-powered agent architecture which lets agents observe, plan and reflect. The framework of Qian et al. \cite{qian_communicative_2023} shows how LLM agents exhibit efficient collaborative and self-corrective abilities in a software development scenario. Similarily, \cite{huang_benchmarking_2023} conduct experiments on LLM-based research agents which perform classical machine learning research involving tasks such as data processing, architecture design and model training. Törnberg et al. \cite{tornberg_simulating_2023} use LLM-based agents in different social media settings to simulate human behavior on a spectrum of toxic discourse and constructive interactions.

\cite{park_generative_2023} uses gpt3.5-turbo
\cite{qian_communicative_2023} uses gpt3.5-turbo-16k
\cite{tornberg_simulating_2023} use gpt-3.5
\cite{huang_benchmarking_2023} use gpt-4

Negotiation Settings:
%\cite{guo_gpt_2023}
%\cite{brookins&debacker_2023}
%\cite{fu et al_2023} uses GPT and claude

Evaluation of Agents:
\cite{davidson_evaluating_2024} evaluate chat-bison, claude-2, command, command-light, gpt-4, gpt-3.5 and (Llama2-70B-Chat)
\cite{wang_rolellm_2023} evaluate chatbots based on gpt-3.5-turbo and GLMPro
\cite{liu_agentbench_2023} AgentBench Evaluation Benchmark

Frameworks for developing agents:
- Langchain Agents
- AutoGen
- CrewAI





\section{Experimental Setup}

\subsection{Research Design and Methodology}

Here goes general information, the approach and an explanation of our evaluation methods (quantitative and qualitative).

\subsection{Negotiation Setup}

Here the setup for the negotiations (fixed-length and open-ended).

\subsection{Interview Setup}

Here the setup for evaluation agent.

\subsection{Parsing}

Here info about our parsing heuristic both for conversations and interviews.

\section{Results}

\subsection{Quantitative}

Quantitative Results.

\subsection{Qualitative}

Qualitative Results.

\section{Limitation}

Limitations here.

\section{Conclusion}

Conclusion here.


%
% ---- Bibliography ----
%
% BibTeX users should specify bibliography style 'splncs04'.
% References will then be sorted and formatted in the correct style.
%
\bibliographystyle{splncs04}
% \bibliography{mybibliography}
%
\bibliography{hubsim_references}
\end{document}
